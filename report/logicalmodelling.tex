\section{Logical Modelling}\label{sec:lm}
After defining the conceptual model, we proceed with the logical data model for the Campus Event Management System.
The logical model outlines the structure of the data items and their relationships in a formal way.
For our system, we use the relational model.
\subsection{Relational Model}
In a relational model, data and their relationships are represented through a collection of interlinked tables.
Each table contains rows and columns, where each column represents an attribute of an entity, and each row holds a record.
The relational schema defines how each table (relation) is structured, specifying the table name and a list of attribute names, each tied to a specific domain.

\subsubsection{Tables and Attributes}

\begin{itemize}
    \item \textbf{User} \\
        \texttt{user(username, username, email, password, user\_type,contact\_number)} \\
        \textbf{Primary Key:} username \\

    \item \textbf{Student} \\
        \texttt{student(student\_id, enrollment\_date, username)} \\
        \textbf{Primary Key:} student\_id \\
        \textbf{Foreign Key:} username $\rightarrow$ user(username)
    
    \item \textbf{Teacher} \\
        \texttt{teacher(teacher\_id, employment\_date, username)} \\
        \textbf{Primary Key:} teacher\_id \\
        \textbf{Foreign Keys:} \\
        \hspace*{1cm} username $\rightarrow$ user(username)
    
    \item \textbf{Venue} \\
        \texttt{Venue(venue\_id, venue\_name)} \\
        \textbf{Primary Key:} venue\_id
    

    \item \textbf{Location} \\
        \texttt{location(location\_id, location\_name)} \\
        \textbf{Primary Key:} location\_id
        
    \item \textbf{Event\_Category} \\
        \texttt{event\_category(category\_id, category\_name)} \\
        \textbf{Primary Key:} category\_id
        
    \item \textbf{Event} \\
        \texttt{event(event\_id, event\_name, description, event\_date, start\_time, end\_time, status, max\_attendees, category\_id, venue\_id)} \\
        \textbf{Primary Key:} event\_id \\
        \textbf{Foreign Keys:} \\
        \hspace*{1cm} category\_id $\rightarrow$ event\_category(category\_id) \\
        \hspace*{1cm} venue\_id $\rightarrow$ Venue(venue\_id)
        
    \item \textbf{Registers} \\
        \texttt{registers(username, event\_id, registration\_date)} \\
        \textbf{Primary Key:} (username, event\_id) \\
        \textbf{Foreign Keys:} \\
        \hspace*{1cm} username $\rightarrow$ user(username) \\
        \hspace*{1cm} event\_id $\rightarrow$ event(event\_id)
        
    \item \textbf{Organized\_By} \\
        \texttt{organized\_by(username, event\_id)} \\
        \textbf{Primary Key:} (username, event\_id) \\
        \textbf{Foreign Keys:} \\
        \hspace*{1cm} username $\rightarrow$ user(username) \\
        \hspace*{1cm} event\_id $\rightarrow$ event(event\_id)
\end{itemize}

\clearpage


% \section{Logical Modeling}

% After defining the conceptual model, we proceed with the logical data model for the Campus Event Management System.
% The logical model outlines the structure of the data items and their relationships in a formal way.
% For our system, we use the relational model.

% \subsection{Relational Model}

% In a relational model, data and their relationships are represented through a collection of interlinked tables.
% Each table contains rows and columns, where each column represents an attribute of an entity, and each row holds a record.
% The relational schema defines how each table (relation) is structured, specifying the table name and a list of attribute names, each tied to a specific domain.

% \subsection{ER Model to Relational Schema}

% The following steps were used to map our ER model to the relational schema:

% \begin{enumerate}
%     \item Each entity type is converted to a table, with each entity attribute becoming a column.
%     \item A primary key is assigned to each table to uniquely identify each row. The primary key may be a single attribute or a combination of attributes.
%     \item For a many-to-many relationship, a separate relationship table is created. This table has a composite primary key, consisting of the primary keys of the linked entities, and any additional attributes specific to the relationship.
%     \item In a one-to-one relationship, the primary key of either entity is added as a foreign key to the other. For one-to-many or many-to-one relationships, the foreign key from one side is added to the many side.
% \end{enumerate}

% In our database schema for the \texttt{campus\_event} system, all user-related information (such as student or teacher details) is stored in the \texttt{user} table, where \texttt{username} is the primary key. This table includes attributes like \texttt{username}, \texttt{name}, \texttt{email}, \texttt{password}, \texttt{user\_type}, and \texttt{contact\_number}.

% The \texttt{event} table holds details of each campus event, including attributes such as \texttt{event\_id} (primary key), \texttt{event\_name}, \texttt{description}, \texttt{event\_date}, \texttt{start\_time}, \texttt{end\_time}, \texttt{status}, \texttt{max\_attendees},\\
% \texttt{currently\_registered}, \texttt{category\_id}, and \texttt{location\_id}. \texttt{category\_id} and \texttt{location\_id} are foreign keys referencing the \texttt{event\_category} and \texttt{location} tables, respectively.

% Since there is a many-to-many relationship between \texttt{user} and \texttt{event} (as multiple users can register for multiple events), we have a \texttt{registers} table. This table uses a composite primary key made up of \texttt{username} and \texttt{event\_id} to uniquely identify each registration. Additionally, it includes the attribute \texttt{registration\_date} to store the date of each registration.

% The \texttt{organized\_by} table represents the relationship between organizers (users) and events they manage, with \texttt{username} and \texttt{event\_id} as a composite primary key. In this table, \texttt{username} references the primary key in the \texttt{user} table, while \texttt{event\_id} references the primary key in the \texttt{event} table.

% There is also a one-to-many relationship between \texttt{event} and \texttt{user}, where each \texttt{event} can be created by a user with a specific \texttt{username}.

% \subsection{List of Relational Schemas for Our System}

% \begin{itemize}
%     \item \texttt{user(username, name, email, password, user\_type, contact\_number)}
%     \item \texttt{student(student\_id, enrollment\_date, username, department)}
%     \item \texttt{teacher(teacher\_id, employment\_date, username)}
%     \item \texttt{event(event\_id, event\_name, description, event\_date, start\_time, end\_time, status, max\_attendees, currently\_registered, category\_id, location\_id)}
%     \item \texttt{event\_category(category\_id, category\_name)}
%     \item \texttt{location(location\_id, address, description, is\_available)}
%     \item \texttt{registers(username, event\_id, registration\_date)}
%     \item \texttt{organized\_by(username, event\_id)}
% \end{itemize}

% This relational schema provides a structured view of the Campus Event Management System, detailing each table’s attributes and their relationships in the database.
