\section{Requirement Gathering and Analysis}

Requirement analysis in this project involves identifying the data that needs to be stored in the database and understanding how it will be accessed by different users. The system is being developed after discussing needs with one of the campus event organizers. Ensuring clear and complete requirements before beginning is essential for creating a system that meets the needs of all stakeholders.

\subsection{Process of Requirement Analysis}

\subsubsection{Discussion with an Event Organizer and students}

Discussions with an event organizer form the core of the analysis. Through these conversations, the aim is to identify the problems encountered in organizing campus events, managing attendees, and allocating resources. Below are some key points from the discussion.

\begin{itemize}
    \item \textbf{Question 1:} What are some challenges in managing events on campus?
    
    \textbf{Event Organizer:} We face issues coordinating with multiple departments, handling large attendance, and ensuring resource availability. With a few dozen events every month, things can get disorganized without a systematic approach.
    
    \item \textbf{Question 2:} Are there any other specific problems?
    
    \textbf{Event Organizer:} Yes. Scheduling conflicts in booking rooms and handling last-minute resource requests often arise.
    With multiple events happening simultaneously, managing these details without a proper system becomes overwhelming.
    
    \item \textbf{Question 3:} What issues do participants face?

    \textbf{Student:} Sometimes, event locations change at the last minute, and we don't get notified on time. We also find it challenging to track which events they've signed up for, especially during busy weeks with several overlapping activities.

    \textbf{Event Organizer:} We also don't have a clear way to notify attendees if an event is canceled or rescheduled. It's frustrating when notifications aren't timely.
\end{itemize}

\subsubsection{Understanding the Requirement}

After discussing these points, it was concluded that the system must handle all event-related data automatically, minimizing errors and delays. The system should allow organizers and participants to manage and access information seamlessly and ensure reliable, timely communication of updates.

Basic information requirements identified are:
\begin{itemize}
    \item \textbf{Event Information}: Event name, date, time, location, and organizer.
    \item \textbf{Participant Information}: Name, ID, email, and role in the event (e.g., attendee, speaker).
    \item \textbf{Resource Information}: Type, availability, and quantity of resources.
    \item \textbf{Location Information}: Room details, capacity, and availability for booking.
\end{itemize}

\subsection{Identifying Stakeholders}

Since the system is being developed to address event management challenges faced on campus, the key stakeholders are:

\begin{itemize}
    \item \textbf{Event Organizers} (teachers, students, and staff involved in planning and managing events).
    \item \textbf{Participants} (students, faculty, and guests attending events).
    \item \textbf{Campus Administration} (managing location bookings and maintaining event records).
    \item \textbf{IT Support Staff} (ensuring the smooth operation and maintenance of the system).
\end{itemize}

As the Question , I aim to create a solution that is user-friendly and effective for all stakeholders.

\subsection{System Specification}

After analyzing requirements and identifying stakeholders, a \textbf{web-based system} has been selected for development. The next step is to design a conceptual model, focusing on a high-level view to confirm that it aligns with the specified requirements.

The system will include features for event creation, participant registration, resource management, and real-time notifications, ensuring that all necessary information is accessible and manageable for everyone involved.
